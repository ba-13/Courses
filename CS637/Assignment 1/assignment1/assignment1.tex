\documentclass[a4paper,12pt]{article}

\usepackage{escexam}

%\excludecomment{solution}

%\renewcommand*\ttdefault{cmvtt}
\setlength{\headheight}{85.05003pt}
\begin{document}

%\vspace*{14ex}

\makeheader{1}                                        % examination number (used to set theorem, lemma numbers)
{August 19, 2022}                           % examination date or deadline
{40}                      % total marks
{Homework Assignment 1}             % Minor Quiz 1, Major Quiz 2, End sem, etc

%\begin{tabular}{cl}
%1. & This question paper contains a total of 12 pages (13 sides of paper). Please verify.\\
%2. & Write your name, roll number, department, section on \textbf{every side of every sheet} of this booklet\\
%3. & Write final answers \textbf{neatly} in the given boxes.\\
%%4. & Do not give derivations/elaborate steps unless the question specifically asks you to provide these.
%\end{tabular}


\begin{problem}{}
(20 points) Problem 7 in the Exercises of Chapter 2 in [LS15].

\noindent
[LS15] Edward A. Lee and Sanjit A. Seshia, Introduction to Embedded Systems, A Cyber-Physical Systems Approach, Second Edition, http://LeeSeshia.org, ISBN 978-1-312-42740-2, 2015. \\
\\
\begin{minipage}{1\textwidth}
  \rectangle{\linewidth}{20cm}
  %\ruledrectangle{7}
\end{minipage}
\newpage
\ \\
\begin{minipage}{1\textwidth}
  \rectangle{\linewidth}{24cm}
  %\ruledrectangle{7}
\end{minipage}
\newpage
\ \\
\begin{minipage}{1\textwidth}
  \rectangle{\linewidth}{24cm}
  %\ruledrectangle{7}
\end{minipage}
\newpage
\ \\
\begin{minipage}{1\textwidth}
  \rectangle{\linewidth}{24cm}
  %\ruledrectangle{7}
\end{minipage}
\end{problem}

\newpage

\begin{problem}{}
(10 points) Problem 2 in the Exercises of Chapter 3 in [LS15].

\noindent
[LS15] Edward A. Lee and Sanjit A. Seshia, Introduction to Embedded Systems, A Cyber-Physical Systems Approach, Second Edition, http://LeeSeshia.org, ISBN 978-1-312-42740-2, 2015. \\
\\
\begin{minipage}{1\textwidth}
  \rectangle{\linewidth}{22cm}
  %\ruledrectangle{7}
\end{minipage}
\newpage
\ \\
\begin{minipage}{1\textwidth}
  \rectangle{\linewidth}{24cm}
  %\ruledrectangle{7}
\end{minipage}
\newpage
\ \\
\begin{minipage}{1\textwidth}
  \rectangle{\linewidth}{24cm}
  %\ruledrectangle{7}
\end{minipage}
\newpage
\ \\
\begin{minipage}{1\textwidth}
  \rectangle{\linewidth}{24cm}
  %\ruledrectangle{7}
\end{minipage}
\end{problem}

\newpage

\begin{problem}{}
(10 points) The states of the linearized model of a vehicle steering system represent the lateral deviation of the vehicle from the x-axis and the angle between the vehicle axis and the x-axis. The output of the linearized model is only the first state. Construct a Simulink model for the vehicle steering system with its controller that includes an observer. The dynamics are available in Example 6.4 and Example 7.3 in [AM09]. Apply a sinusoidal signal as the reference trajectory that specifies the desired deviation of the vehicle from the x-axis with time. Plot the output (lateral deviation of the vehicle from the x-axis) with time.

\noindent
\noindent
[AM09] K. J. Astrom and R. M. Murray. Feedback Systems: An Introduction for Scientists and Engineers. Princeton University Press, 2009. \\
http://www.cds.caltech.edu/~murray/books/AM08/pdf/am08-complete\_22Feb09.pdf . \\
\\
\begin{minipage}{1\textwidth}
  \rectangle{\linewidth}{20cm}
  %\ruledrectangle{7}
\end{minipage}
\newpage
\ \\
\begin{minipage}{1\textwidth}
  \rectangle{\linewidth}{24cm}
  %\ruledrectangle{7}
\end{minipage}
\newpage
\ \\
\begin{minipage}{1\textwidth}
  \rectangle{\linewidth}{24cm}
  %\ruledrectangle{7}
\end{minipage}
\newpage
\ \\
\begin{minipage}{1\textwidth}
  \rectangle{\linewidth}{24cm}
  %\ruledrectangle{7}
\end{minipage}
\end{problem}

\end{document}