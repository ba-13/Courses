\begin{Problem}
The second part of the assignment involves calculating the CPI for each benchmark. You should charge each load and store operation a fixed latency of seventy cycles and every other instruction a latency of one cycle.

Tabulate the CPI of the applications in a table.
\end{Problem}

\begin{Solution}
This can be simply calculated by
\begin{align*}
\texttt{CPI} &= \texttt{Memory Type\% * 70 + Rest Type\% * 1} \\
&= \texttt{Memory Type\% * 69 + 1} \numberthis
\end{align*}

\begin{table}[H]
    \centering
    \caption{CPI per application}
    \label{tab:pB:cpi}
    \begin{tabular}{| l | c | c |}
        \hline
        \multirow{2}{*}{Application} & \multirow{2}{*}{\texttt{Memory Type\%}} & \multirow{2}{*}{CPI} \\
        & & \\
        \hline
        400.perlbench & 35.169 & 24.266 \\
        \hline
        401.bzip2 & 37.141 & 25.627 \\
        \hline
        403.gcc & 30.551 & 21.080 \\
        \hline
        429.mcf & 33.930 & 23.412 \\
        \hline
        450.soplex & 29.190 & 20.141 \\
        \hline
        456.hmmer & 38.397 & 26.494 \\
        \hline
        471.omnetpp & 34.461 & 23.778 \\
        \hline
        483.xalancbmk & 33.053 & 22.806 \\
        \hline
    \end{tabular}
\end{table}


\end{Solution}

% Some other code insertions:
% \begin{lstlisting}[style=CStyle]
% #include <stdio.h>
% int main(void)
% {
%    printf("Hello Fibonacci!"); 
% }
% \end{lstlisting}